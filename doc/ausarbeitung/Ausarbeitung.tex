\documentclass[course=erap]{aspdoc}

%%%%%%%%%%%%%%%%%%%%%%%%%%%%%%%%%
%% Ersetzen Sie in den folgenden zwei Zeilen die entsprechenden -Texte-
%% mit den richtigen Werten.
\newcommand{\theNumber}{Parellele Berechnung der Mandelbrotmenge} % Beispiel: A123
\author{Maximilian Frühauf \and Tobias Klasuen \and Florian Lercher \and Niels Mündler}
%%%%%%%%%%%

\title{Abgabe zu Aufgabe \theNumber}
\date{Wintersemester 2018/19}

\begin{document}
\maketitle

\section{Einleitung}
\begin{itemize}
	\item Was ist MPI / OpenMP / SIMD
	\item Didaktische Ziele
	\item Warum die Mandelbrotmenge
\end{itemize}

\section{Problemstellung und Spezifikation}
\begin{itemize}
	\item Fachliche Spezifikation in Anhang
	\item NFRs in Einleitung schreiben
\end{itemize}


\section{Dokumentation der Implementierung}
\begin{itemize}
	\item Sollte größter Teil werden
	\item \begin{enumerate}
		      \item High level overview?
		      \item Wie läuft das auf meinem System?
		      \item Code Dokumentation / Entwicklerdokumentation
	      \end{enumerate}
\end{itemize}

\section{Ergebnisse / Evaluation}
\begin{itemize}
	\item Skalierbarkeitsgraph
	\item Wie gut ist SIMD / OpenMP / MPI / Mischformen?
	\item Frontend Overhead messen
\end{itemize}

\section{Zusammenfassung}
\begin{itemize}
	\item Zusammenfassung
	\item Ausblick
\end{itemize}

\end{document}
