\section{Ergebnisse / Evaluation}
\begin{itemize}
	\item Skalierbarkeitsgraph
	\item Wie gut ist SIMD / OpenMP / MPI / Mischformen?
	\item Frontend Overhead messen
\end{itemize}

%%%%%%%
% http://www.caps.in.tum.de/himmuc/
% Datenerhebenung (+ auf himmuc!) -> Max
% Skalierungs bzgl anzahl an nodes
% Vergleich OpenMP und mehrere MPI pro Node -> Tobi
% Overhead durch Balancer -> Florian
% Overhead durch MPI / Websocket / DrawTiles (konstant)
% Vergleich SIMD / Rohcode -> Niels

% Nutzbarkeit/ HCI -> Max
% vergleich x86 -> Max

Notiz, entdeckt durch das In-Betracht-Ziehen der Leerregionen:

Probleme bei der Verwendung des Rekursiven Lastbalanierers:
Da es stets im Diskreten eine Minimalgröße für die aufgeteilten Regionen gibt,
kann es sein, dass eine Region für die eine hohe Last vorhergesagt wird viele Worker reserviert -
diese jedoch nicht vollständig ausreizen kann, da die Maximalaufteilung schon erreicht wurde.
Durch die dadurch entstehenden Leerregionen von nicht verwendbaren Workern
kann eine suboptimale Aufteilung enstehen, schlechter noch als die des naiven rekursiven Balancierers.

Dies kann abgeschwächt werden, indem eine Region stets nur soviel Workerressourcen erhält,
wie sie maximal auslasten könnte (Fläche/Fläche minimaler Aufteilung)
