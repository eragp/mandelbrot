\section{Einleitung}

Die Leistung und Geschwindigkeit des individuellen Rechenkerns stagniert seit einigen Jahren.
Moderne Computer erlangen einen Großteil ihrer erhöhten Rechenleistung
seit einiger Zeit nur noch durch Parallelisierung.
Diese sollte jedoch geschickt gestaltet werden, um unerwünschte Seiteneffekte
wie Leerlauf zu vermeiden.

% Didaktische Ziele
\subsection{Didaktische Ziele}

Das zugrundeliegende Problem ist, dass bei der Lastaufteilung einer unabhängigen Menge
von Berechnungen in einem Cluster eine fixe Zuordnung von
zu berechnden Bereichen auf Rechenkerne erzeugt wird. Dauert die Bearbeitung eines Bereiches jedoch deutlich kürzer
als diejenige anderer Abschnitte, so verbringt der reservierte Rechenkern die Zeit bis zum Abschluss der anderen Berechnungen
ohne Arbeit und verbraucht Strom und Platz im Idle-Mode. Da die Kerne eines Clusters jedoch darauf ausgelegt sind, ständig zu arbeiten, sollte dieser Zustand vermieden
werden, um Zeit, Kosten und Energie zu sparen.

Dies kann erreicht werden, indem die Einteilung der Rechenbereiche die vorraussichtliche Rechendauer berücksichtigt.
Dazu werden rechenintensive Bereiche verkleinert und umgekehrt Bereiche mit geringerer Rechenlast vergrößert.
Ziel sollte sein, dass alle Knoten für die Bearbeitung in etwa gleich lang brauchen,
sodass die gegenseitige Wartezeit minimiert wird.

Dieses Projekt soll intuitiv vermitteln, dass bei der Aufteilung unabhängiger
Berechnungen auf ein Cluster eine Abschätzung der benötigten Rechenlast die Gesamtrechendauer deutlich verringern kann.
Außerdem soll ersichtlich sein, wie die verwendete Aufteilung bestimmt wird.

% Warum die Mandelbrotmenge
\subsection{Verwendung der Mandelbrotmenge}

Die Mandelbrotmenge ist eine Teilmenge der komplexen Zahlen. 
Um sie zu berechnen wendet man folgende Formel wiederholt auf jede komplexe Zahl $c$ an:
\begin{equation}
    z_{n+1} = z_{n}^2 + c, \quad z_0 = 0
\end{equation}
In der Mandelbrotmenge befinden sich alle $c$, für die der Betrag von $z_n$ für beliebig große $n$ endlich bleibt.
Wenn der Betrag von $z$ nach einer Iteration größer als 2 ist, so strebt $z$ gegen unendlich, das zugehörige $c$ liegt also nicht in der Menge.
Sobald $|z_n| > 2$ kann die Berechnung daher abgebrochen werden.

Um nun für eine beliebige Zahl zu bestimmen, ob diese in der Mandelbrotmenge liegt, müssen
theoretisch unendlich viele Rechenschritte durchgeführt werden. Zur computergestützten Bestimmung
werden die Rechenschritte nach einer bestimmten Iteration abgebrochen die Zahl als in der Menge liegend betrachtet.

Es handelt sich also um eine Berechnung, die sehr zeitaufwändig ist, wobei
die benötigte Zeit durch Erhöhen der Iterationszahl beliebig erhöht werden kann.
Zusätzlich ist die Berechnung für jede einzelne komplexe Zahl unabhängig von
jeder anderen Zahl.

Diese Eigenschaften ermöglichen es, zweierlei Dinge zu kontrollieren:
\begin{itemize}
    \item Die Dauer der Berechnung
    \item Die Aufteilung der Berechnung auf unterschiedliche Rechenkerne
\end{itemize}

Somit kann gesichert werden, dass eine wahrnehmbare Zeit (100-200 ms) zur Berechnung benötigt wird.
Zudem kann die Unterteilung des zu berechnden Raumes frei gewählt werden, sodass
für verschiedenste Aufteilungen die Gesamtrechenzeit visualisiert werden kann.

\subsection{Darstellung der Mandelbrotmenge}

Komplexe Zahlen lassen sich auch grafisch darstellen, indem man sie in ein Koordinatensystem einträgt.
Dabei entspricht die x-Koordinate dem Realteil und die y-Koordinate dem Imaginärteil der Zahl.
Für das Projekt wird ein Ausschnitt des Bildschirmes als zweidimensionale Darstellung des komplexen Raumes
betrachtet und für jeden darin liegenden Punkt die Zugehörigkeit zur Mandelbrotmenge bestimmt.
Dabei wird der Raum jedoch diskretisiert, indem jedem Pixel des Bildschirmes die komplexen Koordinaten $c$
der linken oberen Ecke zugeordnet werden.

Die grafische Darstellung der Mandelbrotmenge wird durch Einfärbung des zu $c$ gehörigen Pixels erhalten.
Die Zahl der benötigten Iterationen bis zum Abbruch der Berechnung bestimmt dabei die Farbe, sodass alle Pixel
innerhalb der Menge und alle Pixel außerhalb jeweils gleichfarbig sind.

Das entstehende Fraktal ist aufgrund seiner Form auch als “Apfelmännchen” bekannt (siehe \autoref{fig:mandelbrot_visualisierung_beispiel}).
Die Menge ist zusammenhängend, jedoch bilden sich an ihren Rändern viele kleine und sehr komplexe Formen, die visuell ansprechend sind. Es eignet sich daher gut, um optisch Interesse am Projekt zu wecken.

\begin{figure}
    \centering
        \includegraphics[width=0.9\linewidth]{img/Mandelbrot_visualization_example.png}
    \caption{Die Mandelbrotmenge, visualisiert in einem Ausschnitt des komplexen Zahlenraumes.}
    \label{fig:mandelbrot_visualisierung_beispiel}
\end{figure}

% Was ist MPI
\subsection{MPI}

Das Message Passing Interface\footnote{\url{https://www.mpi-forum.org/}} ist eine weit verbreitete Spezifikation, für die Kommunikation zwischen unabhängigen Rechenkernen.
Dadurch existieren viele gut funktionierende Umsetzungen in einer Vielzahl von Programmiersprachen.
Für dieses Projekt wichtig ist, dass es echte Parallelisierung mit geringem Overhead ermöglicht.
So können die einzelnen Berechnungen auf jeweils eigenen unabhängigen Rechenkernen laufen und
die Art der Aufteilung erhält größtmögliche Bedeutung.
Die Gestaltung von MPI erlaubt dabei beliebige Zuordnungen, von Kernen auf einem Prozessor bis hin zu unabhängigen Clusterknoten, die lediglich eine SSH-Verbindung besitzen.

% Was ist OpenMP
% wenn verwendet

% Was ist SIMD
% wenn verwendet

% Qualitative Ziele
\subsection{Qualitätsanforderungen}

Die Benutzeroberfläche soll so leicht und intuitiv wie möglich zu bedienen sein. 
Hierbei soll zudem darauf geachtet werden, dass alle Funktionen nur mit einer minimalen 
Anzahl an Mausklicks auszuführen sind und die Oberfläche nicht überladen wird.

Zudem soll das System robust gestaltet werden.
Dies wird durch die Verwendung von Buttons, Listen, Drop-Down Menüs und Slider gewährleistet,
die die Möglichkeit der Eingabe von ungültigen Werten verhindern.

Es sind keine Sicherheitsfeatures (Benutzerauthentisierung, Verschlüsselung) geplant, da keine sensiblen Daten verarbeitet werden und die Anwendung nicht uneingeschränkt über das Internet zugänglich ist.

% Sonstige Einschränkungen
\subsection{Einschränkungen}

Die Benutzeroberfläche soll in einem Webbrowser lauffähig sein, sodass sie auf beliebigen Endgeräten
zugänglich gemacht werden kann.
Um die erwartete Performanzsteigerung an einem echten Beispiel zu demonstrieren,
soll die Berechnung der Mandelbrotmenge parallel auf mehreren Raspberry Pi’s\footnote{Für ein Beispiel, siehe \url{https://www.raspberrypi.org/products/raspberry-pi-3-model-b/}} oder ähnlichen unabhängigen Kleincomputern oder Rechenkernen zum Einsatz kommen soll.

