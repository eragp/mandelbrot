\section{Einleitung}

% Didaktische Ziele
\subsection{Didaktische Ziele}

Die Leistung und Geschwindigkeit individueller Rechenkerns stagniert seit einigen Jahren.
Moderne Computer erlangen einen Großteil ihrer erhöhten Rechenleistung
seit einiger Zeit nur noch durch Parallelisierung.
Da dies stets einen geringen Overhead verursacht werden gut durchdachte Konzepte zur Ausnutzung mehrerer Kerne
immer wichtiger.

Dieses Projekt soll Nutzern ermöglichen intuitiv ein Gefühl dafür
zu gewinnen, welche wichtige Rolle hierbei die Aufteilung der Last zwischen
den Rechenknoten spielt.

% Warum die Mandelbrotmenge
\subsection{Verwendung der Mandelbrotmenge}

Die Mandelbrotmenge ist eine Teilmenge der komplexen Zahlen. 
Um sie zu berechnen wendet man folgende Formel wiederholt auf jede komplexe Zahl $c$ an:
\begin{equation}
    z_{n+1} = z_{n}^2 + c, \quad z_0 = 0
\end{equation}
In der Mandelbrotmenge befinden sich alle $c$, für die der Betrag von $z_n$ für beliebig große $n$ endlich bleibt.
Wenn der Betrag von $z$ nach einer Iteration größer als 2 ist, so strebt $z$ gegen unendlich, das zugehörige $c$ liegt also nicht in der Menge.
Sobald $|z_n| > 2$ kann die Berechnung daher abgebrochen werden.

Um nun für eine beliebige Zahl zu bestimmen, ob diese in der Mandelbrotmenge liegt, müssen
theoretisch unendlich viele Rechenschritte durchgeführt werden. Zur computergestützten Bestimmung
werden die Rechenschritte nach einer bestimmten Iteration abgebrochen die Zahl als in der Menge liegend betrachtet.

Es handelt sich also um eine Berechnung, die sehr zeitaufwändig ist, wobei
die benötigte Zeit durch Erhöhen der Iterationszahl beliebig erhöht werden kann.
Zusätzlich ist die Berechnung für jede einzelne komplexe Zahl unabhängig von
jeder anderen Zahl.

Diese Eigenschaften ermöglichen es, zweierlei Dinge zu kontrollieren:
\begin{itemize}
    \item Die Dauer der Berechnung
    \item Die Aufteilung der Berechnung auf unterschiedliche Rechenkerne
\end{itemize}

Somit kann gesichert werden, dass eine wahrnehmbare Zeit (100-200 ms) zur Berechnung benötigt wird.
Zudem können die Berechnungen frei auf verfügbare Rechenknoten verteilt werde, sodass
für verschiedenste Aufteilungen eine Rechenzeit visualisiert werden kann.

\subsection{Darstellung der Mandelbrotmenge}

Komplexe Zahlen lassen sich auch grafisch darstellen, indem man sie in ein Koordinatensystem einträgt.
Dabei entspricht die x-Koordinate dem Realteil und die y-Koordinate dem Imaginärteil der Zahl.
Für das Projekt wird ein Ausschnitt des Bildschirmes als zweidimensionale Darstellung des komplexen Raumes
betrachtet und für jeden darin liegenden Punkt die Zugehörigkeit zur Mandelbrotmenge bestimmt.
Dabei wird der Raum jedoch diskretisiert, indem jedem Pixel des Bildschirmes die komplexen Koordinaten $c$
der linken oberen Ecke zugeordnet werden.

Die grafische Darstellung der Mandelbrotmenge wird durch Einfärbung des zu $c$ gehörigen Pixels erhalten.
Die Zahl der benötigten Iterationen bis zum Abbruch der Berechnung bestimmt dabei die Farbe, sodass alle Pixel
innerhalb der Menge und alle Pixel außerhalb jeweils gleichfarbig sind.

Das entstehende Fraktal ist aufgrund seiner Form auch als “Apfelmännchen” bekannt (siehe \ref{mandelbrot_visualisierung_beispiel}).
Die Menge ist zusammenhängend, jedoch bilden sich an ihren Rändern viele kleine und sehr komplexe Formen, die visuell ansprechend sind. Es eignet sich daher gut, um optisch Interesse am Projekt zu wecken.

\begin{figure}
    \label{mandelbrot_visualisierung_beispiel}
    \centering
        \includegraphics[width=0.9\linewidth]{img/Mandelbrot_visualization_example.png}
    \caption{Die Mandelbrotmenge, visualisiert in einem Ausschnitt des komplexen Zahlenraumes.}
\end{figure}

% Was ist MPI
\subsection{MPI}

% Was ist OpenMP
% wenn verwendet

% Was ist SIMD
% wenn verwendet