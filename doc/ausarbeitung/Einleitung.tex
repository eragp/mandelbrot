% TODO Überschriften informativ gestalten? => andere Paper anschauen
% Warum Parallelisierung? Heutzutage wichtig? Auf mehreren Prozessoren?
% Wenn möglich auf eine Seite reduzieren
\section{Einleitung}

Die Leistung und Geschwindigkeit des individuellen Rechenkerns stagniert seit einigen Jahren.
Moderne Computer erlangen einen Großteil ihrer erhöhten Rechenleistung
seit einiger Zeit nur noch durch Parallelisierung.
Diese sollte jedoch geschickt gestaltet werden, um unerwünschte Seiteneffekte
wie Leerlauf zu vermeiden.

In diesem Projekt wird die Mandelbrotmenge verwendet, um dem Benutzer die Effekte einer korrekten Parallelisierung zu verdeutlichen.

% Evaluierungskriterien für erzielbare Benefits/Effekte -> Begründung/ Basis für Evaluation gegen Ende
\subsection{Problemstellung und Motivation}

Mathematisch formuliert, gilt es folgendes Problem zu lösen:

\begin{equation}
minimiere\ {max_{i \in N}(t_i)}\ mit \sum_{i \in N} {t_i} = C \leftrightarrow t_1 = ... = t_n = \frac{1}{C}
\end{equation}

Die Gesamtrechenzeit ist äquivalent zur maximalen Rechendauer eines einzelnen Knotens.
Um diese zu minimieren ist ein Ansatz, die Rechendauer aller Knoten anzugleichen.

Anschaulich beschrieben ist das Problem, dass bei der Lastaufteilung einer unabhängigen Menge
von Berechnungen in einem Cluster eine fixe Zuordnung von
zu berechnenden Bereichen auf Rechenkerne erzeugt wird. Dauert die Bearbeitung eines Bereiches jedoch deutlich kürzer
als diejenige anderer Abschnitte, so verbringt der reservierte Rechenkern die Zeit bis zum Abschluss der anderen Berechnungen
ohne Arbeit und verbraucht Strom und Platz im Idle-Mode.
Dieser Zustand sollte vermieden werden um Zeit, Kosten und Energie zu sparen.

Dies kann erreicht werden, indem die Einteilung der Rechenbereiche die vorraussichtliche Rechendauer berücksichtigt.
Dazu werden rechenintensive Bereiche verkleinert und umgekehrt Bereiche mit geringerer Rechenlast vergrößert.
Ziel ist, dass alle Knoten für die Bearbeitung in etwa gleich lang brauchen,
sodass die gegenseitige Wartezeit minimiert wird.
Da nun zu jedem Zeitpunkt möglichst viele Knoten involviert sind wird die Qualität der Paralellisierung deutlich erhöht
und die Maximalrechendauer gesenkt.

% Didaktische Ziele
% Deutlich machen für Enduser/Laien (!) welche Benefits erzielt werden können
\subsection{Didaktische Ziele}

Dieses Projekt soll eine Oberfläche bereitstellen mit der Endnutzer intuitiv erfahren können, wie
eine Abschätzung der benötigten Rechenlast bei der Aufteilung unabhängiger
Berechnungen auf einem Cluster die Gesamtrechendauer deutlich verringern kann.
Außerdem soll ersichtlich sein, wie die verwendete Aufteilung bestimmt wird.

Die Qualität der Parallelisierung durch MPI soll zudem noch mit anderen Parallelisierungskonzepten
wie OpenMP und SIMD verglichen werden.

% Qualitätsanforderungen und Einschränkungen
\subsection{Qualitätsanforderungen und Einschränkungen}

Die Benutzeroberfläche soll so leicht und intuitiv wie möglich zu bedienen sein.
Falsche Eingaben des Benutzers sollen das System nicht beeinträchtigen oder durch entsprechende GUI-Gestaltung ganz vermieden werden.

Die Benutzeroberfläche soll in einem Webbrowser lauffähig sein, sodass sie auf beliebigen Endgeräten
zugänglich gemacht werden kann.
Um die erwartete Performanzsteigerung an einem echten Beispiel zu demonstrieren,
sollen die Berechnungen parallel auf einem Rechencluster stattfinden.

% zugänglich gemacht werden kann.
% Um die erwartete Performanzsteigerung an einem echten Beispiel zu demonstrieren,
% soll die Berechnung der Mandelbrotmenge parallel auf einem Rechencluster lauffähig sein.
% Hierbei wird ein Fokus auf folgende Eigenschaften gelegt:
% \begin{itemize}
% 	\item Alle Funktionen sollen mit einer minimalen
% 	      Anzahl an Mausklicks auszuführen, sowie durch eine
% 	      minimalistisch Designte Benutzeroberfläche erreichbar sein.
% 	\item Zudem soll das System robust gestaltet werden.
% 	      Dies wird durch die Verwendung von Buttons, Listen, Drop-Down Menüs und Slidern gewährleistet,
%           die die Möglichkeit der Eingabe von ungültigen Werten verhindern.
%           %% TODO drop
% 	\item Es sind keine Sicherheitsfeatures (Benutzerauthentisierung oder Verschlüsselung) geplant,
% 	      da keine sensiblen Daten verarbeitet werden und die Anwendung nicht uneingeschränkt über das Internet zugänglich ist.
% \end{itemize}



% % Sonstige Einschränkungen
% \subsection{Einschränkungen}

% Die Benutzeroberfläche soll in einem Webbrowser lauffähig sein, sodass sie auf beliebigen Endgeräten
% zugänglich gemacht werden kann.
% Um die erwartete Performanzsteigerung an einem echten Beispiel zu demonstrieren,
% soll die Berechnung der Mandelbrotmenge parallel auf einem Rechencluster lauffähig sein.

