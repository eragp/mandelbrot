\section{Zusammenfassung und Ausblick}

%%%%%%
% Zusammenfassung Evaluation
%   --> in Evaluation.tex
% Zusammenfassung des Projekts
%	was haben wir gemacht?
%	was haben wir erreicht?
%	was kann man noch machen? --> Ausblick

Ziel des Projektes war es, ein Programm zur parallelen Berechnung der Mandelbrotmenge auf einem Rechnercluster zu erstellen.
Dazu wurden drei unterschiedliche Ansätze der Parallelisierung genutzt:
\begin{itemize}
	\item \textbf{MPI:} Verteilung der Rechenlast auf die Hardwareknoten des Clusters, Kommunikation über Nachrichten zwischen den Knoten
	\item \textbf{OpenMP:} Verteilung der Rechenlast auf die CPU-Kerne eines Knotens, Kommunikation über geteilten Speicher
	\item \textbf{SIMD:} Verteilung der Rechenlast innerhalb eines CPU-Kernes durch Nutzung von Vektorregistern
\end{itemize}

Die Verteilung auf verschiedene Rechenknoten erfordert eine Lastbalancierung, damit die Knoten gleichmäßig ausgelastet werden.
Hier wurden dafür zwei verschiedene Strategien realisiert:
\begin{itemize}
	\item \textbf{Naive Strategie:} Aufteilung in gleich große Bereiche
	\item \textbf{Strategie mit Vorhersage:} Aufteilung in Bereiche mit gleichem Rechenaufwand mithilfe einer Heuristik
\end{itemize}

Weiterhin sollte der Effekt der Parallelisierung für den Nutzer deutlich werden.
Dazu gibt es eine Web-Schnittstelle in der die Mandelbrotmenge optisch ansprechend dargestellt wird und von der Nutzerin erkundet werden kann.
Der Benutzer hat die Möglichkeit die Parallelisierung mittels OpenMP und SIMD an- oder abzuschalten, sowie die Strategien zur Lastbalancierung ändern.
Außerdem werden Rechenzeit und die Zeit in der Knoten untätig waren in Diagrammen angezeigt.
Die Aufteilung der Lastbalancierung kann als Overlay über die Darstellung Mandelbrotmenge gelegt werden.

% Speedup eintragen, verwendete Optimierungen aufzählen
Durch Anwendung aller Optimierungen (Lastbalancierung mit Vorhersage, OpenMP/4 MPI pro Knoten, SIMD) konnte ein insgesamter Speedup von xx gegenüber der Basisversion (naive Lastbalancierung, 1 MPI Prozess pro Knoten) erreicht werden.

%%%%%%
% Ausblick
%   automatische bestimmung der optimalen knotenmenge
%   erweiterung der fraktale/balancer zwecks didaktik allg. stärkerer HDI/Didaktik interaktion fokus
Als eine Weiterführung dieses Projekts könnten andere Fraktale, wie zum Beispiel Julia-Mengen betrachtet werden.
Auch komplexere Fraktale, zum Beispiel Buddhabrot, stellen eine interessante Herausforderung dar.
Weiterhin könnte eine dynamische Variante der Lastbalancierung mit sogenanntem \textit{Job-Stealing} realisiert werden.
Außerdem kann die Webschnitstelle so erweitert werden, dass noch mehr Parameter der Berechnung, wie Anzahl an Workern oder maximale Iterationszahl, einstellbar sind.