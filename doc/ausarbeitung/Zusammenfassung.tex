\section{Zusammenfassung und Ausblick}

%%%%%%
% Zusammenfassung Evaluation
%   --> in Evaluation.tex
% Zusammenfassung des Projekts
%	was haben wir gemacht?
%	was haben wir erreicht?
%	was kann man noch machen? --> Ausblick

Ziel des Projektes war es, ein Programm zur parallelen Berechnung der Mandelbrotmenge auf einem Rechnercluster zu erstellen.
Dazu wurden drei unterschiedliche Ansätze der Parallelisierung genutzt:
\begin{itemize}
	\item \textbf{MPI:} Verteilung der Rechenlast auf die Hardwareknoten des Clusters, Kommunikation über Nachrichten zwischen den Knoten
	\item \textbf{OpenMP:} Verteilung der Rechenlast auf die CPU-Kerne eines Knotens, Kommunikation über geteilten Speicher
	\item \textbf{SIMD:} Verteilung der Rechenlast innerhalb eines CPU-Kernes durch Nutzung von Vektorregistern
\end{itemize}

Die Verteilung auf verschiedene Rechenknoten erfordert eine Lastbalancierung, damit die Knoten gleichmäßig ausgelastet werden.
Hier wurden dafür zwei verschiedene Strategien realisiert:
\begin{itemize}
	\item \textbf{Naive Strategie:} Aufteilung in gleich große Bereiche
	\item \textbf{Strategie mit Vorhersage:} Aufteilung in Bereiche mit gleichem Rechenaufwand mithilfe einer Heuristik
\end{itemize}

Weiterhin sollte der Effekt der Parallelisierung für den Nutzer deutlich werden.
Dazu gibt es eine Web-Schnittstelle in der die Mandelbrotmenge optisch ansprechend dargestellt wird und von der Nutzerin erkundet werden kann.
Der Benutzer hat die Möglichkeit die Parallelisierung mittels OpenMP und SIMD an- oder abzuschalten, sowie die Strategien zur Lastbalancierung ändern.
Außerdem werden Rechenzeit und die Zeit in der Knoten untätig waren in Diagrammen angezeigt.
Die Aufteilung der Lastbalancierung kann als Overlay über die Darstellung Mandelbrotmenge gelegt werden.

%%%% Hier evtl. noch was zur Evaluation --> Wie sehr überschneidet sich das mit Zusammnefassung im Evaluationsteil?
Im Evaluationsteil des Projekts wurde der Effekt der verschiedenen Parallelisierungsarten auf die Berechnungsdauer ausgewertet.
Es stellt sich heraus, dass es auf jeden Fall sinnvoll ist mehrere Worker zur Berechnung einzusetzen, auch wenn der durch einen neuen Worker erzielte Effekt immer geringer wird.
Auch der Einsatz von SIMD ist lohnenswert. % Und OpenMP auch? --> noch nicht klar
Die Lastbalancierung mit Vorhersage ist deutlich besser, als die naive Strategie, trotzdem gibt es Potential für eine noch bessere Lastverteilung.

%%%%%%
% Ausblick
%   automatische bestimmung der optimalen knotenmenge
%   erweiterung der fraktale/balancer zwecks didaktik allg. stärkerer HDI/Didaktik interaktion fokus