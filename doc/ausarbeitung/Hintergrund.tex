\section{Hintergrund}

% Was ist die Mandelbrotmenge
\subsection{Die Mandelbrotmenge}
\begin{figure}
	\centering
	\includegraphics[width=1\linewidth]{img/Einleitung/mandelbrot}
	\caption{Die Mandelbrotmenge, visualisiert in einem Ausschnitt des komplexen Zahlenraumes.}
	\label{fig:mandelbrot_visualisierung_beispiel}
\end{figure}
Die Mandelbrotmenge ist eine Teilmenge der komplexen Zahlen.
Um sie zu berechnen wendet man folgende Formel wiederholt auf eine komplexe Zahl $c \in \mathbb{C}$ an:
\begin{equation}\label{equ:mandelbrot}
	z_{n+1} = z_{n}^2 + c, \quad z_0 = 0
\end{equation}
In der Mandelbrotmenge befinden sich alle solche $c$, für die \( \lim_{n \rightarrow \infty} |z_n| < \infty \).
Wenn nach der \( n \)ten Iteration \( |z_n| > 2 \) ist, so strebt $z$ gegen unendlich, das zugehörige $c$ liegt also nicht in der Menge.
Somit sobald $|z_n| > 2$ die Berechnung daher abgebrochen werden\cite{424331}.

Um nun für eine beliebige Zahl zu bestimmen, ob diese in der Mandelbrotmenge liegt, müssen
theoretisch unendlich viele Rechenschritte durchgeführt werden. Zur computergestützten Bestimmung
werden die Rechenschritte nach einer bestimmten Iteration abgebrochen die Zahl als in der Menge liegend betrachtet.

\paragraph{Darstellung der Mandelbrotmenge}
Eine Komplexe Zahl \( c \in \mathbb{C} \) lässt sich grafisch darstellen, indem man sie in ein zweidimensionales Koordinatensystem einträgt.
Dabei entspricht die x-Koordinate dem Realteil \( Re(c) \) und die y-Koordinate dem Imaginärteil \( Im(c) \).
Für das Projekt wird ein Ausschnitt des Bildschirmes als zweidimensionale Darstellung des komplexen Raumes
betrachtet und für jeden darin liegenden Punkt die Zugehörigkeit zur Mandelbrotmenge bestimmt.
Dabei wird die komplexe Ebene \( \mathbb{C} \) jedoch diskretisiert, indem jedem Pixel des Bildschirmes die komplexen Koordinaten $c$
der linken oberen Ecke zugeordnet werden.

Die grafische Darstellung der Mandelbrotmenge wird durch Einfärbung des zu $c$ gehörigen Pixels erhalten.
Die Zahl der benötigten Iterationen bis zum Abbruch der Berechnung bestimmt dabei die Farbe, sodass alle Pixel
innerhalb der Menge und alle Pixel außerhalb jeweils gleichfarbig sind.

Das entstehende Fraktal ist aufgrund seiner Form auch als “Apfelmännchen” bekannt (siehe \autoref{fig:mandelbrot_visualisierung_beispiel}).
Am Rand der Menge bilden sich viele kleine und sehr komplexe Formen, die visuell ansprechend sind.

\subsection{MPI}
Das Message Passing Interface\footnote{\url{https://www.mpi-forum.org/}} ist eine weit verbreitete Spezifikation, für die Kommunikation zwischen unabhängigen Rechenknoten.
Es basiert wie der Name bereits andeutet auf dem Konzept, Daten explizit auszutauschen und zwischen den Knoten zu senden statt auf einen geteilten Speicher zuzugreifen.
Dadurch existieren viele gut funktionierende Umsetzungen in einer Vielzahl von Programmiersprachen.
Es ermöglicht echte Parallelisierung mit geringem Overhead.
So können die einzelnen Berechnungen auf jeweils eigenen unabhängigen Rechenkernen laufen und
die Art der Aufteilung erhält größtmögliche Bedeutung.
Die Gestaltung von MPI erlaubt dabei beliebige Zuordnungen, von Kernen auf einem Prozessor bis hin zu unabhängigen Clusterknoten, die lediglich eine SSH-Verbindung besitzen.

% Was ist OpenMP -> Tobi
\subsection{OpenMP}

OpenMP\footnote{\url{https://www.openmp.org/wp-content/uploads/openmp-4.5.pdf}} (Open Multi-Processing) ist ein API, das auf die Parallelisierung von Schleifen und Programmabschnitten auf Shared Memory Systemen spezialisiert ist.
Parallel ausführbare Programmteile werden durch eine spezielle Präprozessor Anweisung für die parallele Ausführung in mehreren Threads gekennzeichnet.
So wird mit geringen Umständen Parallelisierung innerhalb eines Mehrkernrechenbausteines
ermöglicht.
Schlüsselkonzept ist dabei, dass Threads erzeugt werden, die Schleifen paralleliseren und dabei auf den geteilten Speicher
der Kerne zugreifen um Daten auszutauschen oder zusammenzutragen.

% Was ist SIMD -> Niels
\subsection{SIMD} \label{par:introduction_simd}
\enquote{Single Instruction, Multiple Data} setzt auf Hardwareebene um, was der Name bereits andeutet:
Eine Instruktion wird auf verschiedene Daten gleichzeitig angewendet.
Bei einem Projekt wie dem Mandelbrot kann diese Prinzip der Parallelisierung auch gut angewendet werden,
da die einzelnen Punkte unabhängig voneinander sind.

