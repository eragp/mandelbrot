\section{Dokumentation der Implementierung}
\begin{itemize}
	\item Sollte größter Teil werden
	\item \begin{enumerate}
		      \item High level overview?
		      \item Wie läuft das auf meinem System?
		      \item Code Dokumentation / Entwicklerdokumentation
	      \end{enumerate}
\end{itemize}

%%%%%%%%
% Highlevel
% ca 1-2 Seiten
% Buzzwords, grober Überriss

%%%%%%%%
% Middle level
% ca 2-3 Seiten
% Skripte, libraries/packages, Setup
% beispielbilder /output
% getestete Systeme (ARM/Raspi, Debian stretch, x86)
% mindeste was in README.md im master liegen sollte
% viel bereits in READMEs der zugehörigen Ordner zusammengefasst
% backend-deployment -> Niels
% frontend-deployment -> Max

%%%%%%%%
% Low level
% was muss ich als entwickler wissen
% tiles, parallelisierung, structs/klassen, struktur des projekts

% Zuordnung klasse -> Dokumentation
% backend
%   *.main.cpp, init -> Niels
%   include Ordner und CMake -> Niels
%   balancer + tests -> Florian
%   mandelbrot -> Florian (außer MandelbrotSIMD -> Niels)
%   actors
%      Worker -> Tobi
%      Host
%         websocket-funktionen -> Niels
%         Rest -> Tobi
%           + Parallelisierungskonzept (+ welche MPI Method, warum)
%   structs/Netzwerk -> Tobi/Niels/Florian
%   dev-environment (was ist docker und wie/warum) -> Max
% frontend
%    connection -> Niels
%    tileDisplay -> Max
%        TileDisplay.ts
%            Prinzip 
%            genauer Ablauf/Funktionen
%               + Shader
%        WorkerLayer.ts
%           + Gruppierung
%        MatrixView.ts + RegionOfInterest.ts
%        Project.ts
%    visualization -> Niels
%    misc -> Max