\section{Dokumentation der Implementierung}
\begin{itemize}
	\item Sollte größter Teil werden
	\item \begin{enumerate}
		      \item High level overview?
		      \item Wie läuft das auf meinem System?
		      \item Code Dokumentation / Entwicklerdokumentation
	      \end{enumerate}
\end{itemize}

%%%%%%%%
% Highlevel
% ca 1-2 Seiten
% Buzzwords, grober Überriss

%%%%%%%%
% Middle level
% ca 2-3 Seiten
% Skripte, libraries/packages, Setup
% beispielbilder /output
% getestete Systeme (ARM/Raspi, Debian stretch, x86)
% mindeste was in README.md im master liegen sollte
% viel bereits in READMEs der zugehörigen Ordner zusammengefasst
% backend-deployment -> Niels
% frontend-deployment -> Max

%%%%%%%%
% Low level
% was muss ich als entwickler wissen
% tiles, parallelisierung, structs/klassen, struktur des projekts

% Zuordnung klasse -> Dokumentation
% backend
%   *.main.cpp, init -> Niels
%   include Ordner und CMake -> Niels
%   balancer + tests -> Florian
%   mandelbrot -> Florian (außer MandelbrotSIMD -> Niels)
%   actors
%      Worker -> Tobi
%      Host
%         websocket-funktionen -> Niels
%         Rest -> Tobi
%           + Parallelisierungskonzept (+ welche MPI Method, warum)
%   structs/Netzwerk -> Tobi/Niels/Florian
%   dev-environment (was ist docker und wie/warum) -> Max
% frontend
%    connection -> Niels
%    tileDisplay -> Max
%        TileDisplay.ts
%            Prinzip 
%            genauer Ablauf/Funktionen
%               + Shader
%        WorkerLayer.ts
%           + Gruppierung
%        MatrixView.ts + RegionOfInterest.ts
%        Project.ts
%    visualization -> Niels
%    misc -> Max

\subsection{Lastbalancierung}
Um die Mandelbrotmenge effizient parallel zu berechnen, muss die Last gleichmäßig auf die Worker verteilt werden.
Die Aufgabe der Lastbalancierung besteht darin zu einer gegebenen Region und einer Anzahl von Workern eine solche Unterteilung in sogenannte Teilregionen zu finden.
Wichtig dabei ist, dass der garantierte Teiler von Höhe und Breite der Teilregionen dem der angeforderten Region entspricht, da es sonst im Frontend zu Schwierigkeiten bei der Darstellungf kommt.
Die Klassenstruktur der Lastbalancierer entspricht dem Strategy-Pattern. So kann der Balancierer zur Laufzeit leicht gewechselt werden und auch die Erweiterung des Projekts um eine weitere Strategie gestaltet sich einfach.
Was dabei genau beachtet werden muss findet sich im Teil \hyperref[lastbalancierung_erweiterung]{Erweiterung}.
\\
Damit die Unterschiede zwischen guter und schlechter Lastverteilung deutlich werden, wurden hier verschiedene Strategien zur Lastbalancierung implementiert.

\subsubsection*{Naive Strategie}

Bei der naiven Strategie (zu finden in den Klassen NaiveBalancer und RecursiveNaiveBalancer) wird versucht den einzelnen Workern etwa gleich große Teilregionen zuzuweisen.
Dies geschieht allerdings ohne Beachtung der eventuell unterschiedlichen Rechenzeiten innerhalb der Teilregionen.
Die naive Strategie wurde hier in einer nicht-rekursiven  und einer rekursiven Variante implementiert.

\paragraph*{}
%non-recursive naive
Zur nicht-rekursiven Aufteilung wird zuerst die Anzahl der zu erstellenden Spalten und Zeilen berechnet.
Dazu wird der größte Teiler der Anzahl der Worker bestimmt. Dieser gibt die Anzahl der Spalten an, das Ergebnis der Division ist die Anzahl der Zeilen.
Damit ist sichergestellt, dass die Region in die richtige Menge von Teilregionen unterteilt wird.

Als nächstes wird die Breite und Höhe der Teilregionen berechnet. Hierbei ist wichtig, dass der garantierte Teiler erhalten wird.
Die Breite berechnet sich also durch:
\begin{equation*}
	\frac{region.width}{region.guaranteedDivisor * nodeCount} * region.guaranteedDivisor
\end{equation*}
Dabei ist $nodeCount$ die Anzahl der Worker und $region$ die zu unterteilende Region.
Es ist zu beachten, dass es sich hier um eine Ganzzahldivision handelt, deren Rest angibt, wie viele Teilregionen um $region.guaranteedDivisor$ breiter sind.
Die Höhe berechnet sich analog.

Bevor die eigentliche Aufteilung beginnt werden noch die Deltas für Real- und Imaginärteil bestimmt.
Diese geben an, wei groß der Bereich der komplexen Ebene ist, den ein Pixel der Region überdeckt.
Die Deltas können über Methoden der Klasse Fractal berechnet werden.

Zur Aufteilung wird nun mittels zweier verschachtelter Schleifen über die Zeilen und Spalten iteriert.
Die benötigten Start- und Endpunkt der Teilregionen (also die linke obere Ecke und die rechte untere Ecke auf der komplexen Ebene) können nun mithilfe der Schleifenzähler und der Deltas bestimmt werden.
Zusätzlich wird noch der vertikale und horizontale Offset der Teilregion vom Startpunkt der Eingaberegion abgespeichert. Diese Information ermöglicht es die Region im Frontend einfach anzuzeigen.

Falls die aktuell betrachtete Teiregionen zu den Breiteren und/oder Höheren gehört, müssen alle Werte entsprechend angepasst werden.
Die letzte Teilregion einer Spalte bzw. Zeile wird immer so gewählt, dass sie auf jeden Fall mit dem Rand der Eingaberegion abschließt.

Die Teilregionen werden in einem Ergebnisarray gespeichert, welches dann zurückgegeben wird.

\paragraph*{}
% recursiveNaive

\subsubsection*{Strategie mit Vorhersage}

Bei dieser Strategie (zu finden in den Klassen PredictionBalancer und RecursivePredictionBalancer) basiert die Aufteilung der Region auf einer Vorhersage über die Rechenzeit.
Die Teilregionen werden so gewählt, dass sie, entsprechend der Vorhersage, etwa einen ähnlichen Rechenaufwand haben.

Die Vorhersage (struct Prediction) wird von der Klasse Predicter berechnet.
Dazu wird die Region in einer sehr viel geringeren Auflösung berechnet.
Die benötigte Anzahl an Iterationen wird jeweils pro Kachel (Breite und Höhe sind der garantierte Teiler) abgespeichert.
So wird sichergestellt, dass der garantierte Teiler auch nach der Aufteilung noch gilt, da die Balancierer die Vorhersage Eintrag für Eintrag verarbeiten.
Die Genauigkeit der Vorhersage kann über das Attribut $predictionAccuracy$ gesteuert werden:
\begin{itemize}
	\item $predictionAccuracy > 0$: $(predictionAccuracy)^2$ Pixel werden pro Kachel berechnet. Die Summe der Iterationen für die einzelnen Pixel ergibt die Vorhersage für die Kachel.
	\item $predictionAccuracy < 0$: Für $(predictionAccuracy)^2$ Kacheln wird ein Pixel in der Vorhersage berechnet. Es erhalten also mehrere Kacheln diesselbe Vorhersage.
	\item $predictionAccuracy = 0$: Unzulässig, es wird ein Null-Pointer zurückgegeben.
\end{itemize}
Zusätzlich beinhaltet die Vorhersage die Summen der benötigten Iterationen pro Spalte und Zeile, sowie die Gesamtsumme.
Auch die Deltas für Real- und Imaginärteil der einzelnen Kacheln werden angegeben.

Auch die Strategie mit Vorhersage wurde in einer rekursiven und in einer nicht-rekursiven Variante implementiert.

\paragraph*{}
% non-recursive prediction

\paragraph*{}
% recursive prediction

\subsubsection*{Erweiterung}
\label{lastbalancierung_erweiterung}