
\section{Anhang}

\subsection{Detaillierter Start des Backends auf dem HIMMUC}\label{par:detailed_himmuc}

Um Mandelbrot manuell auf dem Host-System zu installieren,
müssen zunächst die notwendigen Bibliotheken installiert werden.
Eine Anleitung dazu findet sich in \autoref{par:himmuc_install_libs}.
Dies muss nur einmal ausgeführt werden, anschließend können die
Programme wie in \autoref{par:himmuc_build_backend} beschrieben kompiliert werden.
Ist dies nicht gewünscht oder erledigt muss das Backend lediglich noch wie
in \autoref{par:himmuc_run_backend} beschrieben gestartet werden.


\paragraph{Lokale Installation der Bibliotheken}\label{par:himmuc_install_libs}


Da hierbei davon ausgegangen wird, dass keine root-Rechte auf dem
Server existieren, werden die Bibliotheken hier lokal in \verb|~/.eragp-mandelbrot|
installiert.
Achten Sie darauf, dass sie Schreibrechte auf dem Ordner haben und
falls sie einen anderen Ordner verwenden wollen,
ersetzen sie jedes Vorkommen des Pfades durch ihren Pfad (insbesondere in der Datei \verb|CMakeLists.txt|).
Die MPI-Bibliothek ist auf dem HimMUC Cluster bereits vorinstalliert
und muss daher nicht mehr aufgesetzt werden.
\begin{figure}[h!]
	\begin{lstlisting}[language=bash, caption={Erstellen des Installationsordners}]
mkdir ~/.eragp-mandelbrot
    \end{lstlisting}
\end{figure}

Die 'Header-only' Libraries \verb|websocketpp| und \verb|rapidjson| müssen
lediglich an einen fixen Ort kopiert werden.
Dies erledigen die Befehle aus \autoref{shell:install_header_libs}.

\begin{figure}[h!]
	\begin{lstlisting}[language=bash, caption={Lokale Installation der Bibliotheken \texttt{websocketpp} und \texttt{rapidjson}.}, label={shell:install_header_libs}]
mkdir "~/.eragp-mandelbrot/install"
cd "~/.eragp-mandelbrot/install"
# Installation von websocketpp
git clone --branch 0.7.0 https://github.com/zaphoyd/websocketpp.git websocketpp --depth 1
# Installation von rapidjson
git clone https://github.com/Tencent/rapidjson/
    \end{lstlisting}
\end{figure}


Aus der Bibliothek \verb|boost| muss die Teilbibliothek \verb|boost_system| lokal kompiliert werden
Dazu werden die Befehle aus \autoref{shell:install_boost} ausgeführt,
um die Version 1.67.0 herunterzuladen, zu entpacken und lokal zu installieren.
Beachten Sie, dass das kompilieren auch wie in \autoref{par:himmuc_build_backend} beschrieben
von einem der Boards ausgeführt werden muss.

\begin{figure}[h!]
    \begin{lstlisting}[language=bash, caption={Lokale Installation der Bibliothek boost.}, label={shell:install_boost}]
# Erstellen der notwendigen Ordnerstrukturen
mkdir "~/.eragp-mandelbrot/install"
mkdir "~/.eragp-mandelbrot/local"
# Einrichten des Internetproxys
export http_proxy=proxy.in.tum.de:8080
export https_proxy=proxy.in.tum.de:8080
# Herunterladen und kompilieren der Boost-Bibliothek
cd "~/.eragp-mandelbrot/install"
wget "https://dl.bintray.com/boostorg/release/1.67.0/source/boost_1_67_0.tar.bz2"
tar --bzip2 -xf boost_1_67_0.tar.bz2
cd boost_1_67_0
./bootstrap.sh --prefix="$HOME/.eragp-mandelbrot/local/" --with-libraries=system
./b2 install
    \end{lstlisting}
\end{figure}

\paragraph{Kompilieren des Backends}\label{par:himmuc_build_backend}

Stellen Sie zunächst sicher, dass auf dem Cluster die Quelldateien des Backendes (im Ordner \verb|backend/|) liegen
(zum Beispiel über \verb|rsync| oder indem sie das Repository dort auch klonen)
Zum Kompilieren des Backends sollte sich auf einen Raspberry Pi oder ODroid
per ssh eingeloggt werden\footnote{Es existiert ein Entwicklerzugang zu einem geteilten Raspberry Pi über die Adresse \url{sshgate-gepasp.in.tum.de}. Dieser wird auch vom Pythonskript genutzt}

Auf dem Board, aus dem Ordner des Backendquellcodes müssen Sie zum kompilieren
des Backendes die Befehle aus \autoref{shell:himmuc_compile_backend} ausführen.

\begin{figure}[h!]
	\begin{lstlisting}[language=bash, caption={Kompilieren des Backends}, label={shell:himmuc_compile_backend}]
# Erstellen und betreten eines build Ordners
mkdir build
cd build
# Aktivieren der MPI Bibliothek
module load mpi
# Kompilieren
cmake ..
make
    \end{lstlisting}
\end{figure}


\paragraph{Ausführen des Backends}\label{par:himmuc_run_backend}

Um das Backend auf dem HimMUC Cluster laufen zu lassen, muss sich zunächst darauf per ssh eingeloggt werden.
Damit für das Frontend kein Unterschied dazwischen besteht, ob das Backend im Dockercontainer , oder
auf einem externen Server ausgeführt wird, ist bei der ssh-Verbindung der Port 9002
des \verb|himmuc.caps.in.tum.de|-Servers an den lokalen Port 9002 gebunden.
So ist das Backend stets unter \url{localhost:9002} verfügbar.
Der zugehörige Befehl zum Login lautet demnach:

\begin{lstlisting}[language=bash]
ssh <rechnerkennung>@himmuc.caps.in.tum.de -L localhost:9002:localhost:9002
\end{lstlisting}

Anschließend muss aus dem Ordner, in dem die ausführbaren Dateien liegen,
für gewöhnlich also der \verb|~/.eragp-mandelbrot/build/| Ordner,
folgender Befehl ausgeführt werden:

\begin{lstlisting}[language=bash]
srun -p <odr|rpi> -n <number of workers+1> -N <number of nodes/raspis> -l --multi-prog <path to eragp-mandelbrot/backend>/himmuc/run.conf &
ssh -L 0.0.0.0:9002:localhost:9002 -fN -M -S .tunnel.ssh <odr|rpi><host number>
\end{lstlisting}

Dabei bestimmt \verb|-n| die Anzahl der laufenden Prozesse (Also Hostprozess und Workerprozesse)
und \verb|-N| die Anzahl zu verwendender Rechenknoten.
Damit anschließend noch alle Anfragen an den Websocketserver auf dem Hostknoten weitergeleitet werden,
muss noch der Port 9002 des \url{himmuc.in.caps.tum.de}-Servers an den Port 9002 des Rechenknotens gebunden werden,
auf dem der Hostprozess läuft.
Der korrekte Knoten ist dabei der Ausgabe des \verb|srun|-Befehles zu entnehmen.
Eine beispielhafte Ausgabe ist in \autoref{shell:himmuc_running_backend_example} zu sehen.

\begin{figure}[h]
	\begin{lstlisting}[language=bash, caption={Beispielhafter Start des Backends. Hierbei ist der Knoten des Hostprozesses \texttt{rpi03}.}, label={shell:himmuc_running_backend_example}]
muendler@vmschulz8:~/eragp-mandelbrot/backend/build$ srun -N4 -n5 -l --multi-prog ../himmuc/run.conf
srun: error: Could not find executable worker
4: Worker: 4 of 5 on node rpi06
2: Worker: 2 of 5 on node rpi04
3: Worker: 3 of 5 on node rpi05
0: Host: 0 of 5 on node rpi03
0: Host init 5
1: Worker: 1 of 5 on node rpi03
0: Core 1 ready!
1: Worker 1 is ready to receive Data.
2: Worker 2 is ready to receive Data.
0: Listening for connections on to websocket server on 9002
0: Core 2 ready!
3: Worker 3 is ready to receive Data.
0: Core 3 ready!
4: Worker 4 is ready to receive Data.
0: Core 4 ready!
muendler@vmschulz8:~/eragp-mandelbrot/backend/build$ ssh ssh -L 0.0.0.0:9002:localhost:9002 -fN -M -S .tunnel.ssh rpi03
    \end{lstlisting}
\end{figure}

\paragraph{Stoppend des Backends}

Um das Backend wieder zu stoppen, müssen der ssh-Tunnel zur Verbindung der Ports
und der \verb|srun|-Prozess gestoppt werden.
Letzterer lässt sich nach dem dämonisieren im vorigen Aufruf nur über die Prozess-ID finden.
Diese zeigt das Tool \verb|ps| an.
\begin{lstlisting}[language=bash]
ssh -S .tunnel.ssh -O exit rpi<host number>
# To stop the node allocation
ps -eo comm,pid | grep srun
kill <srun pid>
\end{lstlisting}