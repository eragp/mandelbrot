%%%%%%%%
% Middle level
% ca 2-3 Seiten
% Skripte, libraries/packages, Setup
% beispielbilder /output
% getestete Systeme (ARM/Raspi, Debian stretch, x86)
% mindeste was in README.md im master liegen sollte
% viel bereits in READMEs der zugehörigen Ordner zusammengefasst
% backend-deployment -> Niels
% frontend-deployment -> Max
\section{Installation der Anwendung}

Um das System zu installieren, muss das Repository mit \verb|git|\footnote{\url{https://git-scm.com/}} lokal geklont werden. Dabei werden die Quelldateien für
das Front- sowie Backend heruntergeladen.

\begin{figure}[h!]
	\begin{lstlisting}[language=bash, caption={Klonen des Repositorys}]
git clone https://gitlab.lrz.de/lrr-tum/students/eragp-mandelbrot.git
        \end{lstlisting}
\end{figure}

\subsection{Lokales Backend}
Eine lokale Installation des Backends zu Entwicklungszwecken ist durch einen Docker\footnote{\url{https://www.docker.com/}} Container möglich.
Dieser bietet eine ähnliche Umgebung zu der des Clusters und ermöglicht schnellere Feedbackzyklen.


Das \verb|run_docker.sh| Skript lädt das benötigte Basis Image, welches alle benötigen Bibliotheken bereits enthält, herunter und erstellt basierend darauf
den Entwicklungscontainer. In diesen werden dann die aktuellen Quelldateien hinein kopiert und kompiliert, wonach das Backend mit Adresse
\verb|ws://localhost:9002| gestartet wird.

\subsection{Backend auf HimMUC Cluster}

\begin{quotation}
	[Der] HimMUC ist ein flexibler Cluster von ARM-Geräten, bestehend aus 40 Raspberry Pi 3 sowie 40 ODroid C2 Single-Board-Computers (SBC).\footnote{\url{http://www.caps.in.tum.de/himmuc/}}
\end{quotation}

\paragraph{Schnellstart}

Um das Programm auf dem HimMUC Cluster zu starten, wurde ein
Python Skript erstellt, das alle notwendigen Schritte übernimmt.
Es führt die Befehle aus \autoref{par:detailed_himmuc} aus, es kann daher bei Problemen zur Fehlerbehebung herangezogen werden.

Stellen sie zunächst sicher, dass sie ein Konto mit Zugangsberechtigungen auf dem HimMUC Cluster besitzen.
Um den eigenen Quellcode auf dem Cluster zu kompilieren muss für die korrekte Funktionsweise des Skriptes zudem
ihr SSH-Key auf dem Cluster abgelegt sein\footnote{siehe \href{https://www.ssh.com/ssh/copy-id}{\texttt{ssh-copy-id}}}.

Außerdem sollten folgende Programme lokal installiert sein:
\begin{itemize}
	\item \verb|rsync|
	\item \verb|ssh|
	\item \verb|python3| (3.5 oder neuer)
\end{itemize}

Starten sie anschließend aus dem Ordner \verb|backend/| den Befehl aus \autoref{shell:start_himmuc}

\begin{figure}[h!]
	\begin{lstlisting}[language=bash, caption={Start des Backends auf dem HimMUC}, label={shell:start_himmuc}]
./run_himmuc.sh <Rechnerkennung> <Anzahl Prozesse> <Anzahl Rechenknoten> 
    \end{lstlisting}
\end{figure}

Das Ergebnis wird ähnlich zu \autoref{shell:start_himmuc_example} aussehen.
Details zu weiteren Optionen des Skripts sind via \verb|--help| verfügbar.
Für eine detailliertere Beschreibung der Installation auf dem \enquote{HimMUC} Cluster,
siehe \autoref{par:detailed_himmuc}

\begin{figure}
	\begin{lstlisting}[language=bash, caption={Beispielausgabe bei Start der Entwicklungsumbegung auf dem HimMUC}, label={shell:start_himmuc_example}]
$ eragp-mandelbrot/backend$ python3 himmuc/start_himmuc.py muendler 10 9
Uploading backend...  sending incremental file list
done
Start mandelbrot with 1 host and 9 workers on 9 nodes... started mandelbrot
Search host node... srun: error: Could not find executable worker
odr00 found
Establish port 9002 forwarding to host node odr00:9002 ... established
System running. Websocket connection to backend is now available at
	ws://himmuc.caps.in.tum.de:9002
Copy and paste this link for an autoconfigured frontend
	http://localhost:3000/?backend=ws://himmuc.caps.in.tum.de:9002
Press enter (in doubt, twice) to stop - do *not* ctrl-c
# Backend output
# Enter

Stopping port forwarding... stopped (-9)
Stopping mandelbrot host and workers... stopped (-9)
    \end{lstlisting}
\end{figure}

\subsection{Installation des Frontends}
Das Frontend ist in TypeScript\footnote{\url{https://www.typescriptlang.org/}} (Erweiterung von JavaScript\footnote{\url{https://en.wikipedia.org/wiki/JavaScript}})
geschrieben und kann somit auf einem beliebigen Endgerät mit einem modernen Webbrowser ausgeführt werden.
Um eine Version lokal zu starten, muss die Paketverwaltung npm\footnote{\url{https://www.npmjs.com/}} installiert werden. Diese verwaltet alle
für das Frontend benötigten Bibliotheken und installiert diese lokal.

Das Kommando \verb|npm start| startet dabei einen lokalen WebServer, welcher eine kompilierte Version des Frontends
unter der Adresse \verb|http://localhost:3000| anbietet. Danach wird der Standardwebbrowser des Systems verwendet, um diese
URL zu öffnen.

Das Frontend verbindet sich automatisch mit der lokalen Adresse \url{ws://localhost:9002}.
Dies kann über den Get-Parameter \verb|ws| in der URL auf die Adresse des eigenen Backends gesetzt werden.