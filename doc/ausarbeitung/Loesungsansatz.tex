%%%%%%%%
% Highlevel
% ca 1-2 Seiten
% Buzzwords, grober Überriss

% Hier wird erklärt, mit welchem Ansatz an das Projekt herangegangen wurde
\section{Lösungsansatz}

% Warum wurde sich für die Darstellung der Mandelbrotmenge entschieden
\subsection{Verwendung der Mandelbrotmenge}

Es handelt bei der Bestimmung der Mandelbrotmenge um eine rechenintensive Operation, wobei
die benötigte Zeit durch Erhöhen der Iterationszahl beliebig erhöht werden kann.
Zusätzlich ist die Berechnung für jede einzelne komplexe Zahl unabhängig von
jeder anderen Zahl.

Diese Eigenschaften ermöglichen es zweierlei Dinge vollständig frei zu kontrollieren:
\begin{itemize}
	\item Die Dauer der Berechnung
	\item Die Aufteilung der Berechnung auf unterschiedliche Rechenkerne
\end{itemize}

Somit kann gesichert werden, dass eine wahrnehmbare Zeit (100-200 ms) zur Berechnung benötigt wird.
Zudem kann die Unterteilung des zu berechnden Raumes frei gewählt werden, sodass
für verschiedenste Aufteilungen die Gesamtrechenzeit visualisiert werden kann.

Als Projekt, das an nicht technisch versierten Anwendern spricht auch der ästethische
Faktor des Mandelbrotfraktals für eine Verwendung.
Er kann ein erstes Interesse für die Anwendung wecken.

\subsection{HimMUC als Rechencluster}
Als Cluster unabhängiger Rechenknoten wurde freundlicherweise vom Lehrstuhl für Rechnerarchitektur und Verteilte System der TU München
das HimMUC Cluster\footnote{\url{http://www.caps.in.tum.de/himmuc/}} bereitgestellt.
Es besteht aus jeweils 40 Raspberry Pi B+\footnote{Für ein Beispiel, siehe \url{https://www.raspberrypi.org/products/raspberry-pi-3-model-b/}} und ODroids.

% Wie ist die (grobe) Architektur des Programmes (parallelisiertes Backend, Oberfläche)
\subsection{Architekturübersicht}
% TODO Ein bisschen mehr details
Das Problem fordert eine Unterteilung in drei wesentliche Bausteine, wie sie in \autoref{fig:architekturuebersicht} zu sehen sind.
Eine Benutzeroberfläche in einem Web Browser des Benutzers (“Frontend”), welches die Benutzerinteraktionen entgegennimmt und mit dem Backend kommuniziert.
Ein Backend-Host übernimmt dazu die Kommunikationsfunktion, verwaltet die eingehenden Rechenaufträge und verteilt sie an die Backend-Worker.
Das Ergebnis dieser Berechnung sendet der Host dann an das Frontend zurück.
Die Backend-Worker nehmen die zugewiesenen Rechenaufträge entgegen und führen die eigentlichen Berechnungen verteilt aus.

\begin{figure}
	% Bindet das als PDF exportierte pptx ein -> vektorgrafik
	% -> pptx bearbeiten statt pdf
	\centering
	\includegraphics[width=0.98\linewidth]{img/Implementierung/Kommunikation.pdf}
	\caption{Architekturübersicht}
	\label{fig:architekturuebersicht}
\end{figure}

\subsection{Lastbalancierung}\label{sec:load_balancing_concepts}

Um die Mandelbrotmenge effizient parallel zu berechnen, sollte die Rechenlast möglichst gleichmäßig auf die Worker verteilt werden.
Die Aufgabe der Lastbalancierung besteht darin zu einer gegebenen Region und einer Anzahl von Workern eine solche Unterteilung in sogenannte Teilregionen zu finden.
Damit die Unterschiede zwischen guter und schlechter Lastverteilung deutlich werden, stehen in diesem Projekt verschiedene Strategien der Lastbalancierung zur Wahl.
Die Strategien lassen sich zur Laufzeit austauschen, um einen direkten Vergleich zu ermöglichen.

\subsubsection{Naive Strategie}

Bei der naiven Strategie wird versucht den einzelnen Workern etwa gleich große Teilregionen zuzuweisen.
Dies geschieht allerdings ohne Beachtung der eventuell unterschiedlichen Rechenzeiten innerhalb der Teilregionen.
Es kann bei Verwendung dieser Stratgegie also durchaus passieren, dass ein Worker noch rechnet, während alle anderen bereits fertig sind.

\subsubsection{Strategie mit Vorhersage}

Bei dieser Strategie basiert die Aufteilung der Region auf einer Vorhersage über die Rechenzeit.
Die Teilregionen werden so gewählt, dass sie, entsprechend der Vorhersage, etwa einen ähnlichen Rechenaufwand haben.
Die optimale Rechenlast für einen Worker berechnet sich also durch:

\begin{equation}\label{equ:desiredN}
	\frac{Gesamtrechenlast}{AnzahlWorker}
\end{equation}

Wenn die Vorhersage hinreichend exakt ist, kann dieser Wert gut angenähert werden.
Dadurch wird die Last gleichmäßiger auf die Worker verteilt als bei der naiven Strategie.

\paragraph*{Bestimmung der Vorhersage}

Die Zugehörigkeit eines Punktes zur Mandelbrotmenge wird nach \autoref{equ:mandelbrot} iterativ berechnet.
Deshalb kann die Anzahl der benötigten Iterationen als Abschätzung der Rechenzeit für diesen Punkt verwendet werden.
Zur Anstellung der Vorhersage wird also die angeforderte Region in deutlich geringerer Auflösung berechnet.
Dies ist eine gute Annäherung an die tatsächlich Rechendauer, da benachbarte Punkte meist eine ähnliche Anzahl an Iterationen benötigen.
Einzig am Rand der Mandelbrotmenge kommt es zu Ungenauigkeiten, weil dort Punkte innnerhalb und außerhalb der Menge für die Vorhersage zusammenfallen.
Die Genauigkeit der Vorhersage zu erhöhen bedeutet zusätzlichen Rechenaufwand während der Balancierung.
Dieser sollte in einem sinnvollen Verhältnis zum Aufwand der Berechnung der Region selbst stehen.
Es ist also wichtig eine Balance zwischen Güte und Geschwindigkeit der Vorhersage zu finden.

\subsubsection{Implementierungsvarianten}

Sowohl die naive Strategie als auch die Strategie mit Vorhersage lassen sich in zwei Varianten umsetzen.
Man kann hierbei einen ganzheitlichen oder einen rekursiven Ansatz wählen.
Bei ersterem wird die gesamte Region in einem Schritt in die gewünschte Anzahl an Teilregionen geteilt.
Dazu werden Zeilen und Spalten gebildet.
Die Grundidee eines rekursiven Ansatzes ist es das Problem so lange in einfachere Teilprobleme aufzuteilen, bis die Lösung offensichtlich ist (Basisfall).
Hier ist der Basisfall die Aufteilung einer Region auf genau einen Worker.
Um diesen zu erreichen wird die Region solange halbiert, bis genug Teilregionen für jeden Worker entstanden sind.

Wo die Grenzen zwischen den Zeilen und Spalten (nicht-rekursiv) bzw. den Hälften (rekursiv) liegen, wird von der zugrundeliegenden Lastbalancierungsstrategie bestimmt.



% Welche Lastbalancierungsmechanismen wurden umgesetzt?

% Wie wurde die Parallelisierung mit MPI angegangen?

% Wie wurde die Parallelisierung mit SIMD angegangen?
\subsection{Parallelisierung mithilfe von SIMD}

Um das Parallelisierungskonzept mithilfe von SIMD zu erklären,
ist es hilfreich zunächst vor Augen zu führen, wie ein Vektor komplexer Koordinaten ohne SIMD verarbeitet würde.
Dies ist in \autoref{src:MandelbrotVect} zu sehen.
Die Berechnung der einzelnen Kooridnaten bleibt gleich, nur die Abbruchbedingung wird auf alle bearbeiteten Koordinaten erweitert.

Es muss dabei solange weiter iteriert werden, bis für alle Komponenten die Berechnung abgebrochen werden darf.
Hierbei ist es kein Problem mit den abgebrochenen Punkten weiter zu rechnen, sofern die
Iterationszahl nur hochgezählt wird wenn das $z_n$ der Koordinate Betragsmäßig kleiner gleich $2$ ist.
Dies gilt, da alle $|z_{n+i}| > 2$ sofern $|z_n| > 2$ \cite{424331}.

\begin{figure}[h!]
	\lstinputlisting[caption={Bearbeitung eines Vektors komplexer Koordinaten in Pseudocode}, label={src:MandelbrotVect}]{code/mandelbrotvect.pseudo}
\end{figure}

Damit kann die Berechnung relativ simpel via Arm NEON Compiler Intrinsics\footnote{Details im Abschnitt 'Compiler Intrinsics' unter \url{https://developer.arm.com/technologies/neon}} implementiert werden (siehe dazu \autoref{src:MandelbrotSIMD32.cpp}).
% 1 Seite

% Wie wurde die Parallelisierung mit OpenMP angegangen?
